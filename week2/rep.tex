\documentclass{report}
\usepackage{listings}
\title{Report}
\author{Mertcan Susuz}
\date{18 June 2016}

\begin{document}
\maketitle
{\Large \textbf{1. Aim}}

\begin{flushleft}
The aim is assigning a value for each character of the alphabet. When user enters character(s) as an input, program will return the value of this character(s).
\end{flushleft}




{\Large \textbf{2.Python code:}}
\newline
\newline
\textbf{{\large a. Full code.}}


\begin{lstlisting}
from random import randint
letters = ("abcdefghijklmnopqrstuvwxyz")
letters_list = []
new = []
output=[]
for j in range(len(letters)):
  letters_list.append(letters[j])

print(letters_list)

for i in range(len(letters)):
   x = { letters[i] : randint(0,9) }
   new.append(x)

print(new)

user_input = input("Enter a letter:")

for char in user_input:
  index = letters_list.index(char)
  num = new[index].get(char)
  output.append(num)

print(output)
\end{lstlisting}


\begin{flushleft}
\textbf{{\large b. Explanation}}
\end{flushleft}

\begin{flushleft}
Firstly, I created a tuple which contains all alphabet letters and created three empty list.
\end{flushleft}
\begin{lstlisting}
from random import randint
letters = ("abcdefghijklmnopqrstuvwxyz")
letters_list = []
new = []
output=[]
\end{lstlisting}

\newpage

\begin{flushleft}
Then I wrote a for loop which each letter separately adds to the list. After that process, there was a list with all the letters and I checked it.
\end{flushleft}

\begin{lstlisting}
for j in range(len(letters)):
  letters_list.append(letters[j])
print(letters_list)
\end{lstlisting}



\begin{flushleft}
I wrote one more for loop which gives me numbers randomly between 0 and 9 and assign these number to each letters.When I do that, I use dictionary for determine the keys and values.In this case, keys are letters and values are numbers.
\end{flushleft}

\begin{lstlisting}
for i in range(len(letters)):
   x = { letters[i] : randint(0,9) }
   new.append(x)
print(new)
\end{lstlisting}


\begin{flushleft}
Finally, I wrote an input code which wants an enter from user. After that, I created a for loop which checks the user's entry, creating an index and a number who represent this this index -letter- , and then append these numbers to the output. When user enter letters, the output will return numbers which represent by these letters.
\end{flushleft}

\begin{lstlisting}
user_input = input("Enter a letter:")

for char in user_input:
  index = letters_list.index(char)
  num = new[index].get(char)
  output.append(num)

print(output)
\end{lstlisting}


\newpage

{\Large \textbf{3.Java Code:}}
\newline
\newline
\newline
{\large \textbf{a. Full Code}}

\begin{small}
\begin{lstlisting}
import java.util.Scanner;
import java.util.List;
import java.util.ArrayList;
import java.util.Arrays;
import java.util.Collection;
import java.util.HashMap;
import java.util.Set;

public class asd {

	public static void main(String[] args) {
				
		HashMap hm = new HashMap();		
		String[] letters = {"a","b","c","d","e","f","g","h","i","j",
		"k","l","m","n","o","p","q","r","s","t","u","v","w","x","y","z"};
		
		for (int i = 0 ; i < letters.length ; i++) {
			hm.put(letters[i],(int)(Math.random()* 10));  }

		Set veriler = hm.entrySet();
		Set anahtarlar = hm.keySet();
		Collection degerler = hm.values();

		System.out.println("Datas : " + veriler);
		System.out.println("Keys : " + anahtarlar);
		System.out.println("Values : " + degerler);
		
		List<String> list = new ArrayList<String>(anahtarlar);
		List<String> values = new ArrayList<String>(degerler);

		Scanner x = new Scanner(System.in);
		System.out.println("Enter a character: ");
		char c = x.next().charAt(0);
		
		if (list.contains(String.valueOf(c))) {
			System.out.println("yes");
			System.out.println("character : " + c +
			" and the value is : " + hm.get(String.valueOf(c)));
		}
		else
			System.out.println("Enter a valid charachter!");
	}

}
\end{lstlisting}
\end{small}


{\large \textbf{b. Explanation}}
\newline
\newline
\begin{flushleft}
I used hashMap class while I writing this code in Java, because this class, hold a value for each key. Firstly, I created an array which contains all alphabet letters, then with using for loop, I assigned a number to each letter, and I put these letters and numbers to the map. In this case, I obtained keys and values together in a map.

\end{flushleft}

\begin{lstlisting}
HashMap hm = new HashMap();
		
String[] letters = {"a","b","c","d","e","f","g","h","i","j","k",
"l","m","n","o","p","q","r","s","t","u","v","w","x","y","z"};
		
for (int i = 0 ; i < letters.length ; i++) {
	hm.put(letters[i],(int)(Math.random()* 10));
}

Set veriler = hm.entrySet();
Set anahtarlar = hm.keySet();
Collection degerler = hm.values();

System.out.println("Datas : " + veriler);
System.out.println("Keys : " + anahtarlar);
System.out.println("Values : " + degerler);
\end{lstlisting}


\begin{flushleft}
Secondly, I convert these set's to the arraylist's in order to do some processes on these lists.
\end{flushleft}
\begin{lstlisting}
List<String> list = new ArrayList<String>(anahtarlar);
List<String> values = new ArrayList<String>(degerler);
\end{lstlisting}

\newpage
\begin{flushleft}
Then, I used scanner method which allows the user to enter variable, and with an if statement, I check whether the entered value is equal to any letter of the alphabet. If it's "yes" then it returns the entered variable -key- , with it's value.
\end{flushleft}
\begin{lstlisting}
Scanner x = new Scanner(System.in);
System.out.println("Enter a character: ");
char c = x.next().charAt(0);
		
if (list.contains(String.valueOf(c))) {
	System.out.println("yes");
	System.out.println("character : " + c +
	" and the value is : " + hm.get(String.valueOf(c)));
		}
else
	System.out.println("Enter a valid charachter!");
\end{lstlisting}

\end{document}